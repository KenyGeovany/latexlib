% LATEX PROJECTS OF REFERENCE

% HEAD OPTIONS: select one or paste it in the main.latex
%\documentclass[11pt,a4paper]{article}
%\documentclass[10pt,spaish]{amsart}

% LANGUAGE
% Set the font (input) encoding
% \usepackage[utf8]{inputenc} % To add accents (\'a) and letter Ñ (\~n). Is no longer required (since 2018).
% Set the font (output) encoding
\usepackage[T1]{fontenc}
% Specific commands
% Adjust the key words and labels for example "Chapter" a "Capítulo". Select multiple lenguages with \selectlanguage{}, this command is open.
% Warning: english must be before spanish for unexpected formating in the cover and bibliography.
\usepackage[english, spanish]{babel} 

\spanishdecimal{.} % Usar punto como separador decimal en vez de coma. Is incompatible con [russian]{babel}. Necessary.

% MATH PACKAGES
\usepackage{amsmath} % Para el entorno matemático. Esencial.
\usepackage{amsfonts} % Para usar \Bbb y \mathbb con R,Q,C etc.
\usepackage{amssymb} % Para simbolos especiales.
\usepackage{amsthm} % Herramientas para definición de teoremas, lemas, etc.
\usepackage{mathtools} % Proporciona mejoras y extensiones para el entorno matemático en comparación con el paquete estándar amsmath.(lo carga automáticamente). Author: Lars Madsen.
\usepackage{accents} % 04.06.2024 Para escrbir un acento encima de un símbolo de congruencia o equivalencia con \accentset{}.

% COMENTARIES (21.07.2024)
\usepackage{comment} % For multiline comentaries.

% ARROW DIAGRAMS (CATEGORY THEORY) AND KNOTS
\usepackage[all,knot,cmtip]{xy}

% MARGINS AND ORIENTATION
% Change the orientation of the page.
\usepackage{lscape}
% Adjust the margins.
\usepackage{geometry}
% Style A
\newcommand{\geometryA}{\geometry{left=2cm, right=2cm, top=2cm, bottom=2cm}}
% Style B
\newcommand{\geometryB}{\geometry{margin=2cm}}
% Choose a style
\makeatletter
\@ifclassloaded{beamer}{
}{  
    % Set
    \geometryA
}
\makeatother


% CUSTOMIZED ENUMERATE PAGES [TESTED]
\usepackage{fancyhdr} % Personaliza los encabezados y pies de página.
\setlength{\headheight}{12.9pt} % Solves the problem in overleaf: Package fancyhdr Warning: \headheight is too small (12.0pt).
% If documentclass is not beamer then import lastpage.
\makeatletter 
\@ifclassloaded{beamer}{
}{
    % This separation is necessary.
    \usepackage{lastpage} % Obtiene automáticamente el número total de páginas.
    \pagestyle{fancy} % Para las funciones de fancyhdr como cfoot, rfoot, etc.
    \cfoot{} % Limpiar el centro del pie de página
    \rfoot{página \thepage{} de \pageref{LastPage}} % Número de página a la derecha
} % (24.08.2024)
% (24.08.2024)

% IMAGES
\usepackage{graphicx} % Allows the use of command includegraphics{}
% Specify the path to your image folder. Paste in main.tex.
% \graphicspath{{./images/}}

% TABLES
\usepackage{float} % To use [H]
\usepackage{multirow} % To use \multirow{nrows}{width}{text}
%\usepackage{booktabs} % SIAM format for tables. Obs. El SIAM tiene macros de latex!

% ADVANCED TYPOGRAPHY
% For text
\usepackage{calligra} % Tipo de fuente cursiva. Use \calligra.
% For math text
\usepackage{mathrsfs} % Latras estilizadas en diagonal. Use \mathscr{}. 
\usepackage{euscript} % Letras estilizadas rectas. Use \EuScript{}.
%%\usepackage{tgcursor} % Tipo de fuente monoespaciada basada en la fuente Courier. Es parte de la colección de fuentes TeX Gyre. Use \texttt{}. No es compatible con beamer, use \usepackage{lmodern}.
\usepackage{MnSymbol} % Modifica el estilo de los símbolos matemáticos.

% COLOR
% If documentclass is not beamer then import lastpage.
\makeatletter 
\@ifpackageloaded{xcolor}{
}{
    % This separation is necessary.
    \usepackage[table,xcdraw]{xcolor} % Color font.
} % (24.08.2024)
\makeatother
\begin{comment} % (24.08.2024)
- Command: definecolor{}{}{}: Defines custom color.
- Command: textcolor{}{}: Changes the color of text.
- With begin{table}: use \rowcolor[HTML]{E0E0E0} which fills a row.
- xcdraw: works with tikz, to use \fill[xcdraw] neverteless \draw is better.
\end{comment}

% ENUMERATIONS AND LISTS
% Redefinition of label format in numbered list: 1), 2), etc.
\renewcommand{\labelenumi}{{\arabic{enumi}}}

% TEXT VERBATIM
\usepackage{verbatim} % It seems not necessary.
\usepackage{fancyvrb} % Used to display literal text that must respect the exact formatting, including spaces and line breaks. Use with: \begin{Verbatim}, \VerbatimInput{}.


% BIBLIOGRAPHY. Select one.
% 1. BIBLATEX
\makeatletter 
\@ifclassloaded{beamer}{%
\usepackage[style=authoryear, sorting=ynt, defernumbers=true, backend=biber, useprefix=true]{biblatex}
\setbeamertemplate{bibliography item}{%
  \ifboolexpr{ test {\ifentrytype{book}} or test {\ifentrytype{mvbook}}
    or test {\ifentrytype{collection}} or test {\ifentrytype{mvcollection}}
    or test {\ifentrytype{reference}} or test {\ifentrytype{mvreference}} }
    {\setbeamertemplate{bibliography item}[book]}
    {\ifentrytype{online}
       {\setbeamertemplate{bibliography item}[online]}
       {\setbeamertemplate{bibliography item}[article]}}%
  \usebeamertemplate{bibliography item}}
}{%
\usepackage[style=numeric, citestyle=numeric-comp,sorting=ynt, defernumbers=true, backend=biber]{biblatex}  % (24.08.2024)
}
\makeatother
\usepackage{csquotes} % Using babel or polyglossia with biblatex, loading csquotes is recommended to ensure that quoted texts are typeset according to the rules of your main language.  % (24.08.2024)
\renewcommand*{\bibfont}{\scriptsize} % Change size of text.
\begin{comment} 
- Compilation order: pdflatex mydocument.tex -> biber mydocument -> pdflatex mydocument.tex -> pdflatex mydocument.tex
- In preamble: \addbibresource{referencias.bib}
- Cite with: parentcite{} or cite{}
- At the end of the body document: \printbibliography
- Print all cites (including non referenced): \nocite{*} 
- Uspackege options: 
    * natbib=true: Enables commands of natbib.
    * style=numeric: Defines the cites as numbers in brackets.
    * citestyle=numeric-comp: Enables cites with numeric and compressed.
    * sorting=ynt: Sorter the cites by Year-Name-Title.
    * defernumbers=true: Delays the assignement of reference numbers until the references are sorted, for coherence.
    * backend=biber: Use biber as backend.
- This package is run using the program biber instead of bibtex. For more information see
    * https://tex.stackexchange.com/questions/406205/no-citation-bibdata-or-bibstyle-command
    * https://tex.stackexchange.com/questions/25701/bibtex-vs-biber-and-biblatex-vs-natbib/299286
\end{comment}

% 2. NATBIB
% Load natbib with options for enumerate, order and compress cites in text: "1, 2, 3" -> "1-3".
%% \usepackage[sort&compress]{natbib}
% Select a style: plain, unsrt, abbrv, acm, alpha, apalike, ieeetr, siam.
%% \bibliographystyle{unsrt}
\begin{comment}
- You cite with: citep{} or cite{}
- At the end of \document: \bibliography{referencias}{}
- Print all cites (including non referenced): \nocite{*} 
\end{comment}

% ALGORITHM Select one.
% ALGORITHM2E
\usepackage[ruled,vlined]{algorithm2e}
\begin{comment}
To write algorithms with a stylized format. 
- ruled: Adds a horizontal ruler at the top and bottom of the algorithm.
- vlined: Adds vertical lines in the algorithm blocks.
- Use begin{algorithm}.
- For comments: out of environment \SetKwComment{Comment}{/* }{ */}, into the environment \Comment*[r]{}
\end{comment}

% ALGPSEUDOCODE
%\usepackage{algpseudocode} % Use \begin{algorithmic}.
%\usepackage{algorithm} % Environment tipe float, table or figure. Use \begin{algorithm}.

% ALGORITHMIC
%\usepackage{algcompatible} % Use \begin{algorithmic}.

% ALGORITHMIC
%\usepackage{algorithmic} % Use \begin{algorithmic}.

\usepackage{listings} % Allows to display source code with syntax highlighting. Use \begin{lstlisting}, \lstinline|| \lstset{}.

% HYPERLINKS
\usepackage{hyperref} % Interactive links, for links, urls, etc.
\hypersetup{
    colorlinks=true,
    linkcolor=blue,
    filecolor=magenta,
    urlcolor=blue,
} % Configuration.
%\usepackage{url} % Paquete para usar URL en el DOI. Hyperref es más completo. (23.06.2024)

% TEXT BOXES (20.07.2024)
\usepackage{tcolorbox}
\newcommand{\widthtcolorbox}{16.5cm} % Custom width of the box. Use \begin{tcolorbox}[colback=white,colframe=black,width=\widthtcolorbox].

% MORE SYMBOLS
\usepackage{siunitx} %Unidades de magintudes físicas.

% SAGEMATH
% \usepackage{sagetex} % Doesn't work in Overleaf.

%-----------------
% New commands
%Para identación de parráfo.
\edef\restoreparindent{\parindent=\the\parindent\relax}
\usepackage{parskip}
\restoreparindent

\usepackage{enumitem}
\usepackage{subcaption}
%\usepackage{etoolbox}
\usepackage{wrapfig} % Para imágenes ajustado al texto.

\makeatletter %(24.08.2024 NO TESTED)
\@ifclassloaded{amsart}{
    \usepackage{amsaddr} %Permite el uso de \address[A1,A2]{UADY \and CIMAT} con formato elegante.
}{}
\makeatother

