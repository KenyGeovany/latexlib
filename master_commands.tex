% Header and footer (18.06.2024)
\fancyhead[L]{} %Clear the left side of the header. It is useful for sobreposition of headers.

% Macro for abstract, keywords and DOI (23.06.2024)
\newcommand{\Abstract}[1]{%
  \begin{abstract}
    #1
  \end{abstract}
}
\newcommand{\Keywords}[1]{%
  \vspace{0.5cm}
  \noindent \textbf{Keywords:} #1
}
\newcommand{\Doi}[1]{%
  \vspace{0.5cm}
  \noindent \textbf{DOI:} \url{#1}
} % Requires package url

%Markers (22.08.2019)
% Define the variable for enumeration style. (06.08.2024)
\newcommand{\theonumber}{section} 
% If documentclass is book, enumerate by chapter.
\makeatletter 
\@ifclassloaded{book}{
  \renewcommand{\theonumber}{chapter}
}{} % (06.08.2024)
% Set environments (see amsthdoc.pdf for more possible commands. 06.08.2024)
\theoremstyle{plain} % THEOREMS.
\newtheorem{theorem}{Theorem}[\theonumber]
\newtheorem{proposition}{Proposition}[\theonumber]
\newtheorem{lemma}{Lemma}[\theonumber]
\newtheorem{corollary}{Corollary}[\theonumber]
\newtheorem{conjecture}{Conjecture}[\theonumber]
\newtheorem{openproblem}{Open Problem}[\theonumber] % (07.08.2024)
\newtheorem{famousproblem}{Famous Problem}[\theonumber] % (07.08.2024)
\theoremstyle{definition} % DEFINITIONS
\newtheorem{axiom}{Axiom}[\theonumber]
\newtheorem{definition}{Definition}[\theonumber]
\newtheorem{convention}{Convention}[\theonumber]
\newtheorem{example}{Example}[\theonumber]
\theoremstyle{remark} % REMARKS 
\newtheorem{note}{Note}[\theonumber]
\newtheorem{remark}{Remark}[\theonumber]
\newtheorem{observation}{Observation}[\theonumber]
\newtheorem{exercise}{Exercise}[\theonumber] %(04.06.2024)
% QED symbol.
%\renewcommand\qedsymbol{$\blacksquare$}

% Custom colors: 23.07.2024
% Is necesary usepackage{xcolor}
% Predefined colors
% red (Rojo), green (Verde claro), blue (Azul fuerte) 
% yellow (Amarillo), cyan (Azul claro), magenta (Rosado) 
% black (Negro), gray (Gris), white (Blanco)
% brown (Café claro), orange (Anaranjado), purple (Rojizo) pink (Rosado palo)
% lime (verde lima), teal (Turqueza), olive (Verde oliva)
\definecolor{custom_red}{rgb}{0.6,0,0}
\definecolor{custom_green}{rgb}{0,0.6,0}
\definecolor{custom_blue}{rgb}{0,0,0.6}
\definecolor{custom_gray}{rgb}{0.35,0.35,0.35} % Gray: the three values are equal.
\definecolor{custom_purple}{rgb}{0.5,0,0.8} % (Morado)


% Date: 14.07.2020
% Tetration and other operators
\DeclareRobustCommand{\triforce}{~\tikz[scale=0.11]\draw[black]
  (1,0) -- +(120:2) --+(1,1.732050808) -- cycle
  (0.5,0.8660254038) -- +(60:1)  --+(360:1)-- cycle
  ;~}
\DeclareRobustCommand{\infrad}{~\tikz[scale=0.16]\draw[black]
  (0.2,0) -- (0.9,0) -- (0.5,0.8660254038) -- (2,0.8660254038) -- (1.6,0) -- (2.3,0)
  ;~}
\DeclareRobustCommand{\suprad}{\rotatebox[origin=c]{180}{$\infrad$}}
\DeclareRobustCommand{\inflog}{~\scriptsize\rotatebox[origin=c]{0}{$\Omega$}~}
\DeclareRobustCommand{\suplog}{~\scriptsize\rotatebox[origin=c]{180}{$\Omega$}~}

%Math functions
%---- Number theory
\DeclareMathOperator{\ord}{ord}
\DeclareMathOperator{\ind}{ind}
%---- Calculus
\DeclareMathOperator{\Div}{div}
\DeclareMathOperator{\Grad}{grad}
%---- Stadistic
\DeclareMathOperator{\Mean}{mean}
%---- Convex analysis
\DeclareMathOperator{\Prox}{prox}
%---- Lineal Algebra
\DeclareMathOperator{\Span}{span}
\DeclareMathOperator{\trace}{trace} % (21.06.2024)
\newcommand{\vect}[1]{\boldsymbol{#1}}
%----- Algebraic Geometry
\DeclareMathOperator{\bfMod}{\textbf{Mod}} % (13.06.2024)
\DeclareMathOperator{\Tor}{Tor} % (13.06.2024)
\DeclareMathOperator{\Dom}{dom}
\DeclareMathOperator{\img}{img}
\DeclareMathOperator{\coker}{coker} % (13.06.2024)
\DeclareMathOperator{\coimg}{coim} % (13.06.2024)
\DeclareMathOperator{\aut}{Aut} % (13.06.2024)
%----- Optimization
\newcommand{\argmin}{\text{arg min}}
%----- General commands

\DeclareMathOperator{\suchthat}{\hspace{1mm}|\hspace{1mm}} %(25.06.2024)

% Macros made by chatgpt for partial derivatives
\newcommand{\der}[3]{\frac{d #1}{d #2} #3}
\newcommand{\derp}[3]{\frac{\partial #1}{\partial #2} #3}
\newcommand{\derpf}[2]{\triangle_{#2}^{+} #1}
\newcommand{\derpb}[2]{\triangle_{#2}^{-} #1}

%Paquete Tikz
\usepackage{tikz}
\usetikzlibrary{shadings, calc}


% Fast comands for mathbb and mathcal (16.07.2024)
\newcommand{\R}{\mathbb{R}} % Classic manner (22.06.2024)
\def\do#1{\csdef{#1}{\mathbb{#1}}} % Fancy maner (22.06.2024)
\docsvlist{N,Z,Q,C,D,T}
% Add fast commands for mathcal letters (16.07.2024)
\def\do#1{\csdef{#1#1}{\mathcal{#1}}}
\docsvlist{A,B,C,D,E,F,G,H,I,J,K,L,M,N,O,P,Q,R,S,T,U,V,W,X,Y,Z}



%\algdef{SE}[DOWHILE]{Do}{doWhile}{\algorithmicdo}[1]{\algorithmicwhile\ #1}%
%\renewcommand{\algorithmicrequire}{\textbf{Input:}}  % Use Input in the format of Algorithm
%\renewcommand{\algorithmicensure}{\textbf{Output:}} % Use Output in the format of Algorithm

% Requires package mathtools 
\DeclarePairedDelimiter\ceil{\lceil}{\rceil}
\DeclarePairedDelimiter\floor{\lfloor}{\rfloor}

% (07.08.2024)
% Comando para llaves. Funciona mejor que \begin{cases}.
\newcommand{\leftcases}[2]{
    \left\lbrace
    \begin{array}{#1}
        #2
    \end{array}    
    \right.
}

% Crear un mirror para \times con \x